
\documentclass[xcolor=dvipsnames,handout]{beamer}


\usepackage[english]{babel}
\usepackage{csquotes}% Recommended
\usepackage[backend=biber,
  style=authoryear,
   ]{biblatex}
\DeclareLanguageMapping{american}{american-apa}

\addbibresource{\jobname.bib}

\hypersetup{colorlinks,
citecolor=blue,
linkcolor=blue,
menucolor=blue,
filecolor=pink,   
anchorcolor=yellow
}

\usepackage{filecontents}
\begin{filecontents*}{\jobname.bib}
@article{kydland1982time,
  title={Time to build and aggregate fluctuations},
  author={Kydland, Finn E and Prescott, Edward C},
  journal={Econometrica},
  pages={1345--1370},
  year={1982},
  publisher={JSTOR}
}

@incollection{gertler2010financial,
  title={Financial intermediation and credit policy in business cycle analysis},
  author={Gertler, Mark and Kiyotaki, Nobuhiro},
  booktitle={Handbook of monetary economics},
  volume={3},
  pages={547--599},
  year={2010},
  publisher={Elsevier}
}

@article{gertler2011model,
  title={A model of unconventional monetary policy},
  author={Gertler, Mark and Karadi, Peter},
  journal={Journal of monetary Economics},
  volume={58},
  number={1},
  pages={17--34},
  year={2011},
  publisher={Elsevier}
}

@article{long1983real,
  title={Real business cycles},
  author={Long Jr, John B and Plosser, Charles I},
  journal={Journal of Political Economy},
  volume={91},
  number={1},
  pages={39--69},
  year={1983},
  publisher={The University of Chicago Press}
}

@article{zhang2019household,
  title={Household debt, financial intermediation, and monetary policy},
  author={Zhang, Yahong},
  journal={Journal of Macroeconomics},
  volume={59},
  pages={230--257},
  year={2019},
  publisher={Elsevier}
}

@article{lozej2018countercyclical,
  title={Countercyclical capital regulation in a small open economy DSGE model},
  author={Lozej, Matija and Onorante, Luca and Rannenberg, Ansgar},
  year={2018},
  journal={ECB Working Paper}
}

@article{liu2021credit,
  title={Credit expansion, bank liberalization, and structural change in bank asset accounts},
  author={Liu, Keqing and Fan, Qingliang},
  journal={Journal of Economic Dynamics and Control},
  volume={124},
  pages={104066},
  year={2021},
  publisher={Elsevier}
}

@book{gali2015monetary,
  title={Monetary policy, inflation, and the business cycle: an introduction to the new Keynesian framework and its applications},
  author={Gali, Jordi},
  year={2015},
  publisher={Princeton University Press}
}

@book{woodford2011interest,
  title={Interest and prices: Foundations of a theory of monetary policy},
  author={Woodford, Michael},
  year={2011},
  publisher={Princeton University Press}
}

@book{miao2020economic,
  title={Economic dynamics in discrete time},
  author={Miao, Jianjun},
  year={2020},
  publisher={MIT press}
}

@article{smets2007shocks,
  title={Shocks and frictions in US business cycles: A Bayesian DSGE approach},
  author={Smets, Frank and Wouters, Rafael},
  journal={American economic review},
  volume={97},
  number={3},
  pages={586--606},
  year={2007}
}

\end{filecontents*}

\DeclareCiteCommand{\textcite}
 {\boolfalse{cbx:parens}}
  {\usebibmacro{citeindex}%
    \printtext[bibhyperref]{\printnames{labelname}%
      \printtext{ (\printfield{year}\printtext{)}}}}
 {\ifbool{cbx:parens}
  {\bibcloseparen\global\boolfalse{cbx:parens}}
  {}%
 \multicitedelim}
{\usebibmacro{textcite:postnote}}


\setbeamertemplate{navigation symbols}{}
\setbeamertemplate{footline}[frame number]
\DeclareMathOperator{\E}{\mathbb{E}}
\usepackage{tikz}
\usepackage{comment}
\usepackage{pgfplots}
\usepackage{xcolor}
\usetikzlibrary{positioning}
\usetikzlibrary{fit}
\usetikzlibrary{backgrounds}
\usetikzlibrary{calc}
\usetikzlibrary{shapes}
\usetikzlibrary{mindmap}
\usetikzlibrary{patterns}
\usepackage{pifont}
%\usepackage{natbib} 
%\usepackage{bibentry}
\newcommand{\xmark}{\ding{55}}%
\newcommand{\cmark}{\ding{51}}%
\usepackage[skins,theorems]{tcolorbox}
\tcbset{highlight math style={enhanced,
  colframe=red,colback=white,arc=0pt,boxrule=1pt}}

\usetikzlibrary{decorations.text}
\pgfplotsset{compat=1.7}
\mode<presentation>
%\usetheme{}
\usecolortheme[named=Black]{structure}


\usefonttheme{serif}     % Font theme: serif
\usepackage{helvet}     % Font family: Concrete Math
\usepackage{tikz}
\setbeamersize{text margin left=04mm,text margin right=04mm} 

\colorlet{titleleft}{Sepia!200!black}
\colorlet{titleright}{white}

\setbeamercolor*{frametitle}{fg=black}  %Set frametitle color

\makeatletter
\pgfdeclarehorizontalshading[titleleft,titleright]{beamer@frametitleshade}{\paperheight}{%
  color(0pt)=(titleleft);
  color(\paperwidth)=(titleright)}

\newcommand{\tikzmark}[1]{\tikz[overlay,remember picture] \node (#1) {};}
\setbeamertemplate{enumerate items}{(\arabic{enumi})}
\setbeamertemplate{itemize items}[circle]
\setbeamercovered{transparent=15}
\setbeamercovered{invisible}
\setbeamertemplate{sections/subsections in toc}[sections numbered]




\title[\today]{\textbf{Debt Covenants, Investment, and Monetary Policy by Ozgen Ozturk}}
\setbeamercolor{author}{fg=black} 
\author{Discussion by Stelios Tsiaras}
\institute{European University Institute}
%\date{%16 December 2020}
%\color{blue}{Latvijas Banka}}





\begin{document}




\begin{frame}[noframenumbering]
\titlepage 
\end{frame}

\begin{frame}[c]\frametitle {\textbf{Summary}} \label{conclusion}
\begin{itemize}
    \setlength\itemsep{2em}
    \item Real Businesses Cycles 
    \item New Keynesian models \smash{\raisebox{.5\dimexpr\baselineskip+\itemsep+\parskip}{$\left.\rule{0pt}{.5\dimexpr2\baselineskip+3\itemsep+5\parskip}\right\}\text{{\color{blue}{DSGE}} models}$}}
    \item \underline{{\color{blue}{D}}ynamic {\color{blue}{S}}tochastic {\color{blue}{G}}eneral {\color{blue}{E}}quilibrium}
    \item RBC \& NK (plain vanilla) models assume perfect financial markets 

$\rightarrow$ DSGE models with {\color{blue}{financial frictions}} go a step further
\end{itemize}
\end{frame}

\begin{frame}[c]\frametitle {\textbf{Summary}} \label{conclusion}
\begin{itemize}
    \setlength\itemsep{2em}
\item DSGE models are micro-founded 
\item They make assumptions regarding: 
\begin{enumerate}
    \item Preferences (log utility, CRRA, GHH...)
    \item Technology (Cobb Douglas PF, CES...)
    \item Market structure (Complete-incomplete markets, heterogeneity, FF...)
\end{enumerate}

\end{itemize}
\end{frame}


\begin{frame}[c]\frametitle {\textbf{Background Literature}} \label{conclusion}
\begin{itemize}
    \setlength\itemsep{1.5em}
\item {\color{black}{Seminal RBC models}}:  \textcite{kydland1982time} \& \textcite{long1983real}
\item {\color{black}{New Keynesian bible(s)}}:
\begin{itemize}
     \item   \textcite{gali2015monetary} \textit{Monetary Policy, Inflation, and the Business Cycle: An Introduction to the New Keynesian
framework and Its Applications, Second Edition}. Princeton University Press
\item \textcite{woodford2011interest}
\textit{Interest and Prices. Foundations of a Theory of Monetary Policy}. Princeton University Press
\end{itemize} 
\item General DSGE 
\begin{itemize}
    \item \textcite{miao2020economic} \textit{Economic Dynamics in Discrete Time}. MIT press
\end{itemize}
\item State of the art multi-shock and frictions NK-DSGE model: \textcite{smets2007shocks}
\end{itemize}
\end{frame}

\begin{frame}[c]\frametitle {\textbf{RBC to NK to FF DSGE}} \label{conclusion}
\begin{itemize}
   \item RBC: Neoclassical model where agents optimize with rational expectations
   \item New Keynesian environment adds price and/ or wage stickiness
   \item Financial frictions eliminate complete markets
   \begin{itemize}
       \item Sometimes these frictions are very specific, derived from microfounded behavior, while sometimes they are more ad-hoc (reduced form)
   \end{itemize}
\end{itemize}
\end{frame}


\begin{frame}[allowframebreaks]\frametitle {\textbf{References}}
%\bibliographystyle{ecta}
\printbibliography
\end{frame}


\end{document}