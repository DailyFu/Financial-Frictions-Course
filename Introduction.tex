\documentclass[xcolor=dvipsnames,handout]{beamer}
\setbeamertemplate{navigation symbols}{}
\setbeamertemplate{footline}[frame number]
\DeclareMathOperator{\E}{\mathbb{E}}
\usepackage{tikz}
\usepackage{comment}
\usepackage{pgfplots}
\usepackage{xcolor}
\usetikzlibrary{positioning}
\usetikzlibrary{fit}
\usetikzlibrary{backgrounds}
\usetikzlibrary{calc}
\usetikzlibrary{shapes}
\usetikzlibrary{mindmap}
\usetikzlibrary{patterns}
\usepackage{pifont}
\newcommand{\xmark}{\ding{55}}%
\newcommand{\cmark}{\ding{51}}%
\usepackage[skins,theorems]{tcolorbox}
\tcbset{highlight math style={enhanced,
  colframe=red,colback=white,arc=0pt,boxrule=1pt}}

\usetikzlibrary{decorations.text}
\pgfplotsset{compat=1.7}
\mode<presentation>
%\usetheme{}
\usecolortheme[named=Black]{structure}


\usefonttheme{serif}     % Font theme: serif
\usepackage{helvet}     % Font family: Concrete Math
\usepackage{tikz}
\setbeamersize{text margin left=04mm,text margin right=04mm} 

\colorlet{titleleft}{Sepia!200!black}
\colorlet{titleright}{white}

\setbeamercolor*{frametitle}{fg=black}  %Set frametitle color

\makeatletter
\pgfdeclarehorizontalshading[titleleft,titleright]{beamer@frametitleshade}{\paperheight}{%
  color(0pt)=(titleleft);
  color(\paperwidth)=(titleright)}

\newcommand{\tikzmark}[1]{\tikz[overlay,remember picture] \node (#1) {};}
\setbeamertemplate{enumerate items}{(\arabic{enumi})}
\setbeamertemplate{itemize items}[circle]
\setbeamercovered{transparent=15}
\setbeamercovered{invisible}
\setbeamertemplate{sections/subsections in toc}[sections numbered]



\title[\today]{\textbf{Debt Covenants, Investment, and Monetary Policy by Ozgen Ozturk}}
\setbeamercolor{author}{fg=black} 
\author{Discussion by Stelios Tsiaras}
\institute{European University Institute}
%\date{%16 December 2020}
%\color{blue}{Latvijas Banka}}



%-------------------------------------------------------------------------------------------
%-------------------------------------------------------------------------------------------
%-------------------------------------------------------------------------------------------
%============================B E G I N  D O C U M E N T=====================================
%-------------------------------------------------------------------------------------------
%-------------------------------------------------------------------------------------------
%-------------------------------------------------------------------------------------------

\begin{document}




\begin{frame}[noframenumbering]
\titlepage 
\end{frame}

\begin{frame}[t]\frametitle {\textbf{Summary}} \label{conclusion}
\begin{itemize}
    \item Very interesting paper!
    \item {\color{blue}{Main question}}: How MP affects firms conditional on their type of debt agreement (debt covenant)? \pause
    \item Develops a model where firms choose their debt covenant 
    \item Mimics the US data fact according to which firms mostly use a cash flow covenant relative to an asset based
    \item Current version results:
    \begin{itemize}
        \item {\color{blue}{Result 1}}: Relationship between productivity, capital and the choice of the debt covenant
        \item {\color{blue}{Result 2}}: An average firm after its $15^{th}$ period of existence pays off its debt stock and uses exclusively internal funding
    \end{itemize}
\end{itemize}
\end{frame}

\begin{frame}[t]\frametitle {\textbf{The Model}} \label{conclusion}
\begin{itemize}
\item Firms choose their debt covenant given the default terms of each type of agreement
    \item {\color{blue}{Asset based}}: If the firm defaults the lender gains a fraction of its assets
    \item {\color{blue}{Cash-flow based}}: Lender gains a multiple $\phi$ of the firm's cash flow (essentially firms output)
    \item Different levels of productivity, capital and borrowing result to different choices of the debt covenant

\end{itemize}
\end{frame}

\begin{frame}[t]\frametitle {\textbf{Main Comments}} \label{conclusion}

The two contracts essentially differ in their default terms
\begin{enumerate}
\item Is loss of the management rights the same with a loss of a cash-flow multiple?
\item Super {\color{blue}{important parameter}}: the multiple {\color{blue}{$\phi$}} of cash-flow in case of default. How is it (will be)  calibrated?
\item Having {\color{blue}{endogenous borrowing constraints}} is quite typical in the literature. What is determined endogenously here and is emphasized since the introduction?
\item There is no relationship between the {\color{blue}{debt covenant}} and the {\color{blue}{interest rate}}. Banks charge always the risk-free rate. 
    %Relationship with firm's capital or cash-flow and debt covenants?  
\begin{itemize}
    \item In this way any -direct- MP interest rate effect is the same for both debt types and not different according to their balance sheets. Both pay $r^B$
    \item Is this to isolate the debt covenant choice effect?
    \item Of course indirect effects do change due to the debt agreements
    \item In the future you could add a friction between banks and firms
\end{itemize}
\end{enumerate}
\end{frame}


\begin{frame}[t]\frametitle {\textbf{Main Comments II}} \label{conclusion}
\begin{enumerate} 
\item Extension: Borrowing parameters changing {\color{blue}{inversely}} with the firms' capital holdings or cash flow?
\item Is it common to assume that cash-flow equals the output of a firm?
\item Cash flow covenant collateral includes also {\color{blue}{capital}}, the asset in the asset based covenant
\begin{itemize}
    \item What's the {\color{blue}{relationship}} between the two schemes?
    \item Comparative exercise between fraction of capital given in the asset based vs. cash-flow collateral. Lots of it should depend on the parameters
\end{itemize}
\item Do the firms choose the covenant according to the higher ability to {\color{blue}{borrow}} or the lower penalty in case of {\color{blue}{default}}?
%\item Result: high productivity firms prefer cash-flow based due to their ability to generate cash. 
%\begin{itemize}
%    \item Isn't that they also have more capital now so they want to avoid an asset based agreement?
%\end{itemize}
\item Are there empirical studies that firms pay back their debt around their $15^{th}$ period?
\end{enumerate}
\end{frame}

\begin{frame}[t]\frametitle {\textbf{Minor Comments}} \label{conclusion}
\begin{itemize}
\item Bank's problem is a bit confusing
\begin{itemize}
    \item Why households do not hold deposits $D$ directly and have this risk-free bond $\alpha$ that in equilibrium $D=\alpha$? 
    (I had to look at the Appendix for that!)
\end{itemize}
\item Why the households hold firm shares? Doesn't all the lending comes from the banks?
\item Definition of debt covenants and covenant types does not belong in the Micro-level evidence. 
\begin{itemize}
    \item Put this in the introduction, or a small chapter after the introduction. It is really helpful!
\end{itemize}
\item Calibration is a crucial part of the paper. Even though it might not be sophisticated at the moment it should have been included

\end{itemize}
\end{frame}

\begin{frame}[t]\frametitle {\textbf{Overall}} \label{conclusion}
\begin{itemize}
\item It's a very nice paper and the {\color{blue}{contribution}} in the literature is clear
%\item All the comments are things that just to make it better (in my view of course)!
\item Also the empirical part is promising but still incomplete
\item Any data of how this is in Europe?

\end{itemize}
\end{frame}



\end{document}