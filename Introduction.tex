
\documentclass[xcolor=dvipsnames,handout,aspectratio=169]{beamer}


\usepackage{natbib}% Recommended
\usepackage{csquotes}
\usepackage{amssymb}
\usepackage{amsfonts}
\usepackage{amsxtra}
\usepackage{amsthm}
\usepackage{scalefnt}
\usepackage{natbib}

\hypersetup{
    colorlinks,
    citecolor=green,
    linkcolor=red,
    backref=true
}

\let\oldbibitem=\bibitem
\renewcommand{\bibitem}[2][]{\label{#2}\oldbibitem[#1]{#2}}
\let\oldcite=\cite
\renewcommand\cite[1]{\hypersetup{linkcolor=blue} \hyperlink{#1}{\oldcite{#1}}}
\let\oldcitep=\citep
\renewcommand\citep[1]{\hypersetup{linkcolor=blue}\hyperlink{#1}{\oldcitep{#1}}}
\let\oldciteauthor=\citeauthor
\renewcommand\citeauthor[1]{\hypersetup{linkcolor=blue}\hyperlink{#1}{\oldciteauthor{#1}}} 



\setbeamertemplate{navigation symbols}{}
\setbeamertemplate{footline}[frame number]
\DeclareMathOperator{\E}{\mathbb{E}}
\usepackage{tikz}
\usepackage{comment}
\usepackage{pgfplots}
\usepackage{xcolor}
\usetikzlibrary{positioning}
\usetikzlibrary{fit}
\usetikzlibrary{backgrounds}
\usetikzlibrary{calc}
\usetikzlibrary{shapes}
\usetikzlibrary{mindmap}
\usetikzlibrary{patterns}
\usepackage{pifont}
%\usepackage{natbib} 
%\usepackage{bibentry}
\newcommand{\xmark}{\ding{55}}%
\newcommand{\cmark}{\ding{51}}%
\usepackage[skins,theorems]{tcolorbox}
\tcbset{highlight math style={enhanced,
  colframe=red,colback=white,arc=0pt,boxrule=1pt}}

\usetikzlibrary{decorations.text}
\pgfplotsset{compat=1.7}
\mode<presentation>
%\usetheme{}
\usecolortheme[named=Black]{structure}


\usefonttheme{serif}     % Font theme: serif
\usepackage{helvet}     % Font family: Concrete Math
\setbeamersize{text margin left=09mm,text margin right=09mm} 

\colorlet{titleleft}{Sepia!200!black}
\colorlet{titleright}{white}

\setbeamercolor*{frametitle}{fg=black}  %Set frametitle color

\makeatletter
\pgfdeclarehorizontalshading[titleleft,titleright]{beamer@frametitleshade}{\paperheight}{%
  color(0pt)=(titleleft);
  color(\paperwidth)=(titleright)}

\newcommand{\tikzmark}[1]{\tikz[overlay,remember picture] \node (#1) {};}
\setbeamertemplate{enumerate items}{(\arabic{enumi})}
\setbeamertemplate{itemize items}[circle]
\setbeamercovered{transparent=15}
\setbeamercovered{invisible}
\setbeamertemplate{sections/subsections in toc}[sections numbered]




\title[\today]{\textbf{A Course on DSGE Models with Financial Frictions \\ Introduction}}
\setbeamercolor{author}{fg=black} 
\author{Stelios Tsiaras}
\institute{European University Institute}
%\date{%16 December 2020}
%\color{blue}{Latvijas Banka}}

\begin{document}


\begin{frame}[noframenumbering]
\titlepage 
\end{frame}

\begin{frame}[c]\frametitle {\textbf{Summary}} \label{conclusion}
\begin{itemize}
    \setlength\itemsep{2em}
    \item Real Businesses Cycles 
    \item New Keynesian models \smash{\raisebox{.5\dimexpr\baselineskip+\itemsep+\parskip}{$\left.\rule{0pt}{.5\dimexpr2\baselineskip+3\itemsep+5\parskip}\right\}\text{{\color{Magenta}{DSGE}} models}$}}
    \item \underline{{\color{Magenta}{D}}ynamic {\color{Magenta}{S}}tochastic {\color{Magenta}{G}}eneral {\color{Magenta}{E}}quilibrium}
    \item RBC \& NK (plain vanilla) models assume perfect financial markets 

$\rightarrow$ DSGE models with {\color{Magenta}{financial frictions}} go a step further
\end{itemize}
\end{frame}

\begin{frame}[c]\frametitle {\textbf{Summary}} \label{conclusion}
\begin{itemize}
    \setlength\itemsep{2em}
\item DSGE models are micro-founded 
\item They make assumptions regarding: 
\begin{enumerate}
    \item Preferences (log utility, CRRA, GHH...)
    \item Technology (Cobb Douglas PF, CES...)
    \item Market structure (Complete-incomplete markets, heterogeneity, FF...)
\end{enumerate}

\end{itemize}
\end{frame}


\begin{frame}[c]\frametitle {\textbf{Background Literature}} \label{conclusion}
\begin{itemize}
    \setlength\itemsep{1.5em}
\item {\color{black}{Seminal RBC models}}:  \cite{kydland1982time} \& \cite{long1983real}
\item {\color{black}{New Keynesian main textbooks}}:
\begin{itemize}
     \item   \cite{gali2015monetary} \textit{Monetary Policy, Inflation, and the Business Cycle: An Introduction to the New Keynesian
framework and Its Applications, Second Edition}. Princeton University Press
\item \cite{woodford2011interest}
\textit{Interest and Prices. Foundations of a Theory of Monetary Policy}. Princeton University Press
\end{itemize} 
\item General DSGE
\begin{itemize}
    \item \cite{miao2020economic} \textit{Economic Dynamics in Discrete Time}. MIT press
\end{itemize}
\item State of the art multi-shock and frictions NK-DSGE model: \cite{smets2007shocks} 
\end{itemize}
\end{frame}

\begin{frame}[c]\frametitle {\textbf{RBC to NK to FF DSGE}} \label{conclusion}
\begin{itemize}
   \item RBC: Neoclassical model where agents optimize with rational expectations
   \item New Keynesian environment adds price and/ or wage stickiness
   \item Financial frictions eliminate complete markets
   \begin{itemize}
       \item Sometimes these frictions are very specific, derived from micro-founded behaviour, while sometimes they are more ad-hoc (reduced form)
   \end{itemize}
\end{itemize}
\end{frame}

\begin{frame}[c]\frametitle {\textbf{Models' Solution}} \label{conclusion}
\begin{itemize}
        \setlength\itemsep{1.5em}
   \item What {\color{Magenta}{characterizes}} the solution of a DSGE model?
   \begin{itemize}
       \item The {\color{Magenta}{optimality conditions}} obtained through the various maximizations
       \item {\color{Magenta}{Constraints}}
       \item {\color{Magenta}{Shock Processes}}
   \end{itemize}
   \item The steady state is obtained by transforming all equations from dynamic to static
\end{itemize}
\end{frame}

\begin{frame}[c]\frametitle {\textbf{Models' Solution}} \label{conclusion}
\begin{itemize}
        \setlength\itemsep{1.5em}
   \item What {\color{Magenta}{is}} the solution of a DSGE model?
   \begin{itemize}
       \item A set of {\color{Magenta}{policy functions}}
   \end{itemize}
   \item A simple RBC model example
\begin{itemize}
    \item $C_t = g_c(K_{t},Z_t)$ 
    \item $K_t = g_k(K_{t-1},Z_t)$
\end{itemize}
\item Finding the policy function can be a very difficult problem: there is rarely an analytical solution and we therefore use numerical techniques
\item The general idea is to approximate the policy function with a polynomial $\hat{g_c}(K_{t},Z_t) = g_c(K_{t},Z_t)$:
\begin{itemize}
    \item  $\hat{g_c}(K_{t},Z_t) = \alpha_c + \alpha_{c,k} K_t+ \alpha_{c,z} Z_t$
\end{itemize}
\end{itemize}
\end{frame}

\begin{frame}[c]\frametitle {\textbf{Solution Methods}} \label{conclusion}
\begin{itemize}
\item A large number of solution methods have been proposed to solve DSGE models
\begin{itemize}
    \item See \cite{fernandez2016solution} for an almost complete characterization
\end{itemize}
\item {\color{Magenta}{Perturbation}} algorithms build Taylor series approximations to the solution of a DSGE model around its deterministic steady state 
\begin{itemize}
    \item This is what {\color{blue}{Dynare}} does
    \item We will focus on this
\end{itemize}
\item {\color{Magenta}{Projection}} methods handle DSGE models by building a function indexed by some coefficients that approximately solves our set of functions
\begin{itemize}
    \item The coefficients are selected to minimize a residual function that evaluates how far away the solution is from generating a zero 
    \item Example: Parametrized Expectations Algorithm by: \cite{den1990solving}
\end{itemize}
\item Value function iterations 
\end{itemize}
\end{frame}


\bibliographystyle{natbib}
\bibliography{course_bibl.bib} 

\end{document}